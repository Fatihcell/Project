%% Generated by Sphinx.
\def\sphinxdocclass{report}
\documentclass[letterpaper,10pt,english]{sphinxmanual}
\ifdefined\pdfpxdimen
   \let\sphinxpxdimen\pdfpxdimen\else\newdimen\sphinxpxdimen
\fi \sphinxpxdimen=.75bp\relax
\ifdefined\pdfimageresolution
    \pdfimageresolution= \numexpr \dimexpr1in\relax/\sphinxpxdimen\relax
\fi
%% let collapsible pdf bookmarks panel have high depth per default
\PassOptionsToPackage{bookmarksdepth=5}{hyperref}

\PassOptionsToPackage{booktabs}{sphinx}
\PassOptionsToPackage{colorrows}{sphinx}

\PassOptionsToPackage{warn}{textcomp}
\usepackage[utf8]{inputenc}
\ifdefined\DeclareUnicodeCharacter
% support both utf8 and utf8x syntaxes
  \ifdefined\DeclareUnicodeCharacterAsOptional
    \def\sphinxDUC#1{\DeclareUnicodeCharacter{"#1}}
  \else
    \let\sphinxDUC\DeclareUnicodeCharacter
  \fi
  \sphinxDUC{00A0}{\nobreakspace}
  \sphinxDUC{2500}{\sphinxunichar{2500}}
  \sphinxDUC{2502}{\sphinxunichar{2502}}
  \sphinxDUC{2514}{\sphinxunichar{2514}}
  \sphinxDUC{251C}{\sphinxunichar{251C}}
  \sphinxDUC{2572}{\textbackslash}
\fi
\usepackage{cmap}
\usepackage[T1]{fontenc}
\usepackage{amsmath,amssymb,amstext}
\usepackage{babel}



\usepackage{tgtermes}
\usepackage{tgheros}
\renewcommand{\ttdefault}{txtt}



\usepackage[Bjarne]{fncychap}
\usepackage{sphinx}

\fvset{fontsize=auto}
\usepackage{geometry}


% Include hyperref last.
\usepackage{hyperref}
% Fix anchor placement for figures with captions.
\usepackage{hypcap}% it must be loaded after hyperref.
% Set up styles of URL: it should be placed after hyperref.
\urlstyle{same}

\addto\captionsenglish{\renewcommand{\contentsname}{Contents:}}

\usepackage{sphinxmessages}
\setcounter{tocdepth}{1}



\title{Project}
\date{Feb 24, 2025}
\release{v1.00}
\author{Fatih ONAY}
\newcommand{\sphinxlogo}{\vbox{}}
\renewcommand{\releasename}{Release}
\makeindex
\begin{document}

\ifdefined\shorthandoff
  \ifnum\catcode`\=\string=\active\shorthandoff{=}\fi
  \ifnum\catcode`\"=\active\shorthandoff{"}\fi
\fi

\pagestyle{empty}
\sphinxmaketitle
\pagestyle{plain}
\sphinxtableofcontents
\pagestyle{normal}
\phantomsection\label{\detokenize{index::doc}}


\sphinxAtStartPar
Add your content using \sphinxcode{\sphinxupquote{reStructuredText}} syntax. See the
\sphinxhref{https://www.sphinx-doc.org/en/master/usage/restructuredtext/index.html}{reStructuredText}
documentation for details.

\sphinxstepscope


\chapter{Research Proposal}
\label{\detokenize{Project Proposal:research-proposal}}\label{\detokenize{Project Proposal::doc}}
\begin{sphinxadmonition}{note}{Note:}
\sphinxAtStartPar
\sphinxstyleemphasis{This is a demo project proposal for the mobility in Japan.}
\end{sphinxadmonition}


\section{Background of Proposed Research Plan}
\label{\detokenize{Project Proposal:background-of-proposed-research-plan}}
\sphinxAtStartPar
Understanding of the human brain in terms of functional and strutural architecture has been rapidly increasing
due to the advances in imaging and recording technologies such as fMRI, EEG, and MEG. This collective endeavor
of scientists around the world proved that decoding neuronal activity associated with cognitive, attentional, perception
processes is not beyond reach thanks to the integration of computational modeling strategies with bulk of data {[}\hyperlink{cite.index:id2}{MHL+22}{]}.
However, attempting to construct accurate and generalizable models of brain function remains a challenge due to the complexity
of neuronal dynamics, inter and intra\sphinxhyphen{}individual variability, and high\sphinxhyphen{}dimensionality of neuroimaging data. The primary
objective of modelling brain using neuroimaging data is to discover dynamics of large\sphinxhyphen{}scale neuronal activity associated with
cognitive and behavioral processes.

\sphinxAtStartPar
Modeling frameworks are bottom\sphinxhyphen{}up or top\sphinxhyphen{}down which


\section{Purpose of Proposed Research}
\label{\detokenize{Project Proposal:purpose-of-proposed-research}}
\sphinxAtStartPar
Effective  computational models must account for these factors while integrating multimodal data sources to enhance interpretibility and
predictive power.


\section{Proposed Plan}
\label{\detokenize{Project Proposal:proposed-plan}}
\sphinxAtStartPar
Achieving unified generalizable model to simulate wide range of the large scale brain activity requires to choose the
type of modality and
\begin{itemize}
\item {} 
\sphinxAtStartPar
\sphinxstylestrong{Objective 1:} Select and justify the appropriate neuroimaging modality or combination of modalities that best captures large\sphinxhyphen{}scale neuronal dynamics.

\item {} 
\sphinxAtStartPar
\sphinxstylestrong{Objective 2:} Develop a robust preprocessing and feature extraction pipeline to manage high\sphinxhyphen{}dimensional neuroimaging data.

\item {} 
\sphinxAtStartPar
\sphinxstylestrong{Objective 3:} Construct a computational model that integrates multi\sphinxhyphen{}modal data to simulate brain activity associated with cognitive and behavioral processes.

\item {} 
\sphinxAtStartPar
\sphinxstylestrong{Objective 4:} Validate the model using independent datasets and benchmark its performance in predicting functional outcomes.

\item {} 
\sphinxAtStartPar
\sphinxstylestrong{Objective 5:} Assess the generalizability of the model across different populations and cognitive tasks.

\end{itemize}


\section{Expected Results and Impacts}
\label{\detokenize{Project Proposal:expected-results-and-impacts}}
\sphinxstepscope


\chapter{New Section}
\label{\detokenize{Project Development:new-section}}\label{\detokenize{Project Development::doc}}
\begin{sphinxadmonition}{important}{Important:}
\sphinxAtStartPar
Let’s try something
\end{sphinxadmonition}


\section{New Subsection}
\label{\detokenize{Project Development:new-subsection}}
\sphinxAtStartPar
Okey !

\begin{sphinxthebibliography}{MHL+22}
\bibitem[MHL+22]{index:id2}
\sphinxAtStartPar
Ross D Markello, Justine Y Hansen, Zhen\sphinxhyphen{}Qi Liu, Vincent Bazinet, Golia Shafiei, Laura E Suárez, Nadia Blostein, Jakob Seidlitz, Sylvain Baillet, Theodore D Satterthwaite, and others. Neuromaps: structural and functional interpretation of brain maps. \sphinxstyleemphasis{Nature Methods}, 19(11):1472\textendash{}1479, 2022.
\end{sphinxthebibliography}



\renewcommand{\indexname}{Index}
\printindex
\end{document}